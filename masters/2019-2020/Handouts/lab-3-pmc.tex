\documentclass[a4paper,10pt]{article}
\usepackage{fullpage}
\usepackage{times}
\usepackage{url}
\usepackage{hyperref}

\def\UrlBreaks{\do/\do-}

\begin{document}

\title{L41: Lab 3 - Micro-Architectural Implications of IPC}
\author{Dr Robert N.~M. Watson \and Dr Graeme Jenkinson}
\date{2019-2020}
\maketitle

\noindent
The goals of this lab are to:

\begin{itemize}
\item Introduce hardware performance counters (hwpmc).
\item Explore micro-architectural implications of IPC.
\item Gather additional data to support the writing of your first assessed lab
  report.
\end{itemize}

\noindent
You will do this by applying PMC to analyse the behaviour of the same potted,
kernel-intensive IPC benchmark used in \textit{L41: Lab 2 - Kernel
Implications of IPC}.

\section*{Background: Performance Monitoring Counters (PMC)}

Hardware performance counters are a low-level processor facility that gathers
statistics about \textit{architectural} and \textit{micro-architectural}
performance properties of code execution and data access.

\begin{description}
\item[Architectural features] are those exposed explicitly via the documented
instruction set and device interfaces -- e.g., the number of instructions
executed, or the number of load instructions executed.

\item[Micro-architectural features] have to do with the programmer-transparent
implementation details, such as pipelining, superscalar execution, caches, and
so on -- e.g., the number of L2 cache misses taken.
\end{description}

The scope for \textit{programmer transparency} (e.g., what is included in the
architecture vs. micro-architecture) varies by Instruction-Set Architecture
(ISA): whereas MIPS exposes certain pipelining effects to the programmer
(e.g., branch-delay slots), ARM and x86 minimise the visible exposure other
than performance impact.
MIPS and ARM both require explicit cache management by the operating system
during I/O operations and code loading, whereas x86 also masks those
behaviours.

Performance counters can be used in two ways: \textit{counting}, in which
instances of a particular architectural or micro-architectural event are
counted during program execution; and \textit{sampling}, in which $1/n$
instances of the event will trigger a hardware trap that allows, for example,
a stack trace to be taken (similar to historic timer-driven profiling
techniques).
In this lab, we will use PMC only in counting mode.

PMC support may be integrated into the operating system in a variety of ways.
Typically, this is done by an additional tracing and profiling framework: in
FreeBSD, HWPMC; in Linux, OProfile and related tools.
It is also possible to integrate PMC support with DTrace, as has been done in
Solaris, but not yet in FreeBSD.
On FreeBSD, HWPMC provides a programming API that allows applications to
measure their own micro-architectural impacts.
We have integrated explicit PMC support into the L41 IPC benchmark using these
APIs, allowing it to count events such as memory accesses and cache misses at
various points in the cache hierarchy.

The ARM Cortex A8, used in the BeagleBone Black, can track events using up to
four sources at a time; we will typically track the number of cycles, the
number of instructions executed architecturally (i.e., that weren't canceled
in the pipeline due to, for example, a branch mispredict), and then pairs of
counters tracking a particular part of the cache hierarchy.
We will focus almost exclusively on memory-related counters, rather than
looking at other micro-architectural performance events such as branch
prediction.
This is because our IPC benchmark results will be most strongly affected by
memory footprint of our buffers and IPC primitives.

FreeBSD also includes tools to sample PMC behaviour by process or
systemically, capturing stack traces via sampling, and mapping them back to
program symbols or annotated source code.
You may wish to also use these tools to help explain performance behaviour
(i.e., not just that L2 cache misses were dominant at a particular buffer
size, but also that the majority of cache misses were taken in a particular
part of the kernel), but that is not required for this lab.
If you wish to use these tools, please see the FreeBSD \texttt{pmcstat(8)}
man page for details on capturing counter data for whole-program and
whole-system analysis.
Note that, although on the hardware side PMC may have no measurable probe
effect, the software framework around PMC (i.e., to virtualise counters across
multiple processes) can introduce substantial probe effect, which we must be
aware of when using these counters for performance analysis.

\section*{The benchmark}

The IPC benchmark has a \texttt{-P} argument that requests use of performance
counters to analyse the IPC loop.
Performance counters are configured in ``process mode'', meaning that they
track user and kernel events associated with a process and its descendents, so
should include events from all three of our benchmark modes including their
execution in kernel, but not other system events.
Where events occur asynchronously in a kernel thread not explicitly associated
with the user process, those events will not be counted (e.g., kernel work
performed by a timer on behalf of a user process).

\subsection*{Running the benchmark}

Lab 3 employs the same benchmark used in Lab 2.
As before, you can run the benchmark using the \texttt{ipc-static} and
\texttt{ipc-dynamic} commands, specifying various benchmark parameters.
When the new performance-counter argument is used, additional information will
be printed about the processor-level behaviour of the IPC loop.
Do ensure that, as in Lab 2, you have increased the kernel's maximum
socket-buffer size.

\subsection*{Performance-counter arguments}

Performance-counter support are enabled using the \texttt{-P} flag, which
accepts one argument identifying the set of counters to track during
execution.
Due to the 4-counter limit in the Cortex A8, it is not possible to count all
the potential events of interest at the same time.
As such, some care will be required to take multiple samples and consider
counter readings as members of a distribution.
The following counter modes are supported:

\begin{description}
\item[l1d] Track cache hits and refills (L2 loads performed as a result of a
  cache miss) on the L1 data cache.
  This counter may include the effects of speculated, but canceled,
  instructions.
\item[l1i] Track cache refills on the L1 instruction cache; L1 cache hits are
  not directly countable on this processor.
  This counter may include the effects of speculated, but canceled,
  instructions.
\item[l2] Track cache hits on the L2 cache, which is used for both
  instruction and data access.
  L2 cache misses/refills are not directly countable on this processor.
  This counter may include the effects of speculated, but canceled,
  instructions.
\item[mem] Count architecturally originated memory reads and writes: i.e.,
  load and store instructions.
  This counter will not include the effects of speculated, but canceled,
  instructions.
\item[axi] Track memory accesses issued over the AXI bus: i.e., to actual
  DRAM or to perform I/O.
  Note that I/O accesses can be significant -- e.g., network traffic will pass
  over the AXI bus -- so attempt to minimise I/O during the benchmark.
  This counter may include the effects of speculated, but canceled,
  instructions.
\item[tlb] Track refills in the instruction and data Translation Lookaside
  Buffers (TLBs), which cache page-table entries in hardware.
  This counter may include the effects of speculated, but canceled,
  instructions.
\end{description}

\subsection*{Example benchmark commands}

This command instructs the IPC benchmark to capture information on memory
instructions issued when operating on a socket with a 512-byte buffer from a
single thread:

\begin{verbatim}
# ipc/ipc-static -i local -b 512 -P mem 1thread
\end{verbatim}

\noindent
This command performs the same benchmark while tracking L1 data-cache hits and
refills:

\begin{verbatim}
# ipc/ipc-static -i local -b 512 -P l1d 1thread
\end{verbatim}

\noindent
This command performs the same benchmark while tracking L2 cache hits:

\begin{verbatim}
# ipc/ipc-static -i local -b 512 -P l2 1thread
\end{verbatim}

\noindent
And this command performs the same benchmark while tracking memory operations
that make it out the bus to DRAM (or I/O devices):

\begin{verbatim}
# ipc/ipc-static -i local -b 512 -P axi 1thread
\end{verbatim}

\subsection*{Cortex A8 caches}

The ARM Cortex A8 has independent level-1 instruction and data caches (each
32k) and a shared instruction/data level-2 cache (256k).
The cache line size is 64 bytes, and most counters will refer to cache lines
rather than bytes of memory.
For example, the rough utilised memory bandwidth of the system might be
estimated as the sum of AXI reads and writes multiplied by 64, although the
actual data used will depend on how effectively software has been able to pack
data into cache lines.
As we are working with virtually contiguous buffers and most access is via
memory copies, this is a reasonable approximation in our environment.

\subsection*{Performance counters}

The following performance counters are exposed by the IPC benchmark via its
various PMC modes:

\begin{description}

\item[AXI\_READ] The number of AXI-bus read transactions.

\item[AXI\_WRITE] The number of AXI-bus write transactions.

\item[CLOCK\_CYCLES] The number of clock cycles.

\item[DTLB\_REFILL] The number of data-TLB refills.

\item[INSTR\_EXECUTED] The number of instructions executed architecturally.

\item[ITLB\_REFILL] The number of instruction-TLB refills.

\item[L1\_DCACHE\_ACCESS] The number of L1 data-cache hits.

\item[L1\_DCACHE\_REFILL] The number of L1 data-cache refills.

\item[L1\_ICACHE\_REFILL] The number of L1 instruction-cache refills.

\item[L2\_ACCESS] The number of L2 cache hits.

\item[MEM\_READ] The number of memory read instructions that executed
  architecturally.

\item[MEM\_WRITE] The number of memory write instructions that executed
  architecturally.
\end{description}

\section*{Note on graphs in this lab report}

Because of the large amounts of data (and number of data sets) explored in
this lab, you will need to pay significant attention in writing the lab report
to how you present data visually.
Graphs should make visual arguments, and how a set of graphs are plotted can
support (or confuse) that argument.
Make sure all graphs are clearly presented with labels and textual
descriptions helping the reader identify the points you think are important.

When two graphs have the same independent variable (e.g., buffer size), it is
important that they use the same X axis in terms of labelling and scale.
Graphs with the same X axis will often benefit from being arranged so that
they align horizontally on the page, such that inflection points can be
visually compared.

Where an X axis is identical, and dependent variables have the same Y axis
(e.g., both measure bandwidth and have the same scale), placing them on the
same graph is frequently useful, as visual artefacts (such as intersecting
lines, differing slopes) have specific meaning and will pop out at readers.
Be careful to clearly label different lines, and ideally use shading, point
symbol, and/or colour to make the visual distinction clear.
If you have having trouble distinguishing the different data sets, then there
are too many data sets on the graph.

Where an X axis is identical, but dependent variables differ on their scales
(e.g., one measures bandwidth, and a second cache refill rate), placing them on
the same graph could lead to confusion as, for example, line intersections may
not actually have meaning.
You can, however, vertically stack multiple graphs on the same X axis,
allowing inflection points and changes in slopes to be visually compared.
Do this by aligning the X axes of the two graphs, and then `squinching' (a
technical term) the two close together; as the X axes will have identical
units and values, you can have the graphing package include labels only for
the bottom graph.
This will allow comparison of linked data -- e.g., a larger graph showing
bandwidth, and a set of smaller graphs showing micro-architectural effects
such as TLB and cache refill rates, to be visually compared to make it easy to
assess possible correlation.

\section*{Experimental questions (part 2)}

These questions supplement the experimental questions in the Lab 2 handout.
As with the configuration described in the prior handout, they are with
respect to a fixed total IPC size, statically linked version of the benchmark,
and refer only to IPC-loop and not whole-program analysis.
Consider all three IPC types (pipe, socket, socket with \texttt{-s}) in
\texttt{2thread} mode:
% and both \texttt{2thread} and \texttt{2proc} models:

\begin{itemize}
  \item How does changing the IPC buffer size affect the architectural and
    micro-architectural aspects of cache and memory behaviour -- and why?
  \item Can we reach causal conclusions about the scalability of the kernel's
    pipes and local socket implementations given additional evidence from
    processor performance counters?
  \item Explore the impact of the probe effect on your causal investigation;
    how has DTrace changed the behavior of the benchmark?
\end{itemize}

\noindent
Your lab report must address the experimental questions in both Lab 2 and this
lab.

\end{document}
